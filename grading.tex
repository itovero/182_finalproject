\textbf{(TODO - remove before submission)Grading Rubric}

\textbf{20pt - Working implementation} in a new framework (JAX, Julia, Matlab, or Tensorflow if not used in the original code).
All the code that the students implemented works and is self-contained. Includes instructions for setting up the environment and running without any issues. Solutions do not have bugs and graders are able to run the code without any problems.
Students clearly cite any external sources used (for example, if they used something they found on paperswithcode or a blog post, this is fine, as long as it clearly cited and student have created significant work of their own.) 
Students have clearly made a solid effort and gotten their project in working shape. It doesn’t have to be perfect, and it’s fine if the students have a “wish list” of additional things they would do if they had the time.

\textbf{25pt - Content and Correctness}
Coding questions engage with selected key concepts in the paper and are easy to follow and pedagogically useful. Implementation is doable for someone with CS 182 level mathematical maturity (not too difficult, but not completely trivial either).
Any exposition/mathematical background about the concepts as provided in the problem description is correct. 
Code solutions are provided and fully correct. 
If there are analytical questions, the full LaTeX and PDF export of the assignment (and solutions if separate) are provided. LaTeX files compile correctly. The problems and solutions are completely correct and engage with the material in a pedagogically useful way.
Doing the full problem (coding and written parts) would take an average CS 182 student 1.5-2hr to complete. 


\textbf{15pt - Scaffolding}
Project provides necessary scaffolding for a student to engage with the material. Any code that students are expected to implement is explained clearly through the use of text cells in Jupyter and/or code comments.
If the project group uses any external packages, these are provided, or there are simple instructions on how to download them from before starting the assignment
If possible, autograding tests/sanity checks to make sure a student’s implementation is correct are provided. Gradescope autograder files are not needed—simple sanity checks that compare actual values against expected values are enough. 

\textbf{10pt - Readability/Clarity}
HW assignment and commentary are easy to read and follow. Any mathematical notation used is understandable, and any non-standard notation is clearly explained. The assignment and commentary are free of spelling/grammar errors.

