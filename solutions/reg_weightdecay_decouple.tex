\qns{Weight Decoupling, Adaptive Methods, and Finally, AdamW \cite{Loshchilov}}


\href{https://pytorch.org/docs/stable/generated/torch.optim.AdamW.html}{AdamW}, by definition, is an weight-decoupled adaptive stochastic optimization method that implements weight decaying effects onto Adam (ADAptive Moment estimation) through entering in coresponding values for the \texttt{weight\_decay} argument. But if you take a look at PyTorch's documentation on \href{https://pytorch.org/docs/stable/generated/torch.optim.Adam.html}{Adam}, we also see a \texttt{weight\_decay} parameter for us to set - which makes one wonder, "setting this parameter for Adam should be equivalent to doing the same with AdamW - right?". In this problem, we will explore how these two methods are not equivalent and how \texttt{weight\_decay} in Adam is a potential misnomer as $\ell_2$-regularization $\neq$ weight decay in Adam. Along the way, we'll also introduce you to the idea of weight decoupling, fundamentals of adaptive gradient methods, and its relationship with $\ell_2$-regularization and weight decay. Finally, we will convince you that AdamW, a weight decoupled version of Adam, leads to a hyperparameter space that is better behaved and easier to tune.

\begin{enumerate}[(a)]
    %Part A
    \qitem{ \textbf{SGD and Weight Decoupling}
    
    Before taking a look into Adam and AdamW, let's look at Stochastic Gradient Descent (SGD) to familiarize the idea of weight decoupling (SDGW) and how for SGD optimization methods, $\ell_2$ regularization is equivalent to weight decay - \underline{even with momentum}. 
    
    \begin{enumerate}[(i)]
        %Part A (i)
        \qitem{
        Assume that we have a dataset $\{\vec{x}_i, y_i\}_{i = 1}^n$ with data $\vec{x}_i \in \mathbb{R}^d$ and label $y_i \in \mathbb{R}$ and we have access to a model $f_{\vec{\theta}}(\cdot)$ with parameter $\vec{\theta}$. Recall that weight decay is defined as weights (a.k.a. parameters) $\vec{\theta}$ decaying exponentially per time-step $t$ as 
        \begin{equation}\label{weight_decay} \vec{\theta}_{t+1} = (1-\lambda_w)\vec{\theta}_t - \eta \grad L_{\vec{\theta}}(y_i, f_{\vec{\theta}}(\vec{x}_i))\bigg|_{\vec{\theta} = \vec{\theta}_t}
        \end{equation}
        for some weight decaying factor $\lambda_w$, learning rate $\eta$, initial weight value $\vec{\theta}_0$, and loss function $L_{\vec{\theta}}(\cdot, \cdot)$ evaluating on a random data point $(\vec{x}_i, y_i)$. Suppose we define a new loss function $L^{reg}_{\vec{\theta}}(\cdot, \cdot)$ that has an $\ell_2$-regularization term added to our original loss function
        \begin{equation} L_{\vec{\theta}}^{reg}(y_i, f_{\vec{\theta}}(\vec{x}_i))) = L_{\vec{\theta}}(y_i, f_{\vec{\theta}}(\vec{x}_i)) + \frac{\lambda_{\ell_2}}{2} ||\vec{\theta}||^2_2
        \end{equation}
        for some penalization factor $\lambda_{\ell_2}$. \textbf{Show that the update rule of SGD with the $L_{\vec{\theta}}^{reg}(\cdot, \cdot)$ loss function is equivalent to performing weight decay updates with $L_{\vec{\theta}}(\cdot, \cdot)$. Derive the relationship between $\lambda_w$ and $\lambda_{\ell_2}$ in terms of the learning rate $\eta$ and potentially other variables in the update rule.}
        
        \textit{NOTE: Recall that the update rule of SGD under a loss function $L_{\vec{\theta}}(\cdot, \cdot)$ is \begin{equation}\vec{\theta}_{t+1} = \vec{\theta}_t - \eta \grad L_{\vec{\theta}}(y_i, f_{\vec{\theta}}(\vec{x}_i))\bigg|_{\vec{\theta} = \vec{\theta_t}}
        \end{equation}}
        }
        %Solution Part A (i)
        \sol{
            Taking the gradient of $L_{\vec{\theta}}^{reg}(y_i, f_{\vec{\theta}}(\vec{x}_i))$ in respect to $\vec{\theta}$ lets us see that 
            \begin{equation}
            \grad L_{\vec{\theta}}^{reg}(y_i, f_{\vec{\theta}}(\vec{x}_i))\bigg|_{\vec{\theta} = \vec{\theta_t}} = \grad L_{\vec{\theta}}(y_i, f_{\vec{\theta}}(\vec{x}_i))\bigg|_{\vec{\theta} = \vec{\theta_t}} + \lambda_{\ell_2}\vec{\theta}\bigg|_{\vec{\theta} = \vec{\theta_t}}
            \end{equation}
            Plugging in this equation into our update rule of SGD gets us
            \begin{equation}\vec{\theta}_{t+1} = \vec{\theta}_t - \eta (\grad L_{\vec{\theta}}(y_i, f_{\vec{\theta}}(\vec{x}_i))\bigg|_{\vec{\theta} = \vec{\theta_t}} + \lambda_{\ell_2}\vec{\theta_t} ) 
            \end{equation}
            Grouping $\vec{\theta}_t$ gives us
            \begin{equation}\vec{\theta}_{t+1} = (1 - \eta\lambda_{\ell_2})\vec{\theta}_t - \eta (\grad L_{\vec{\theta}}(y_i, f_{\vec{\theta}}(\vec{x}_i))\bigg|_{\vec{\theta} = \vec{\theta_t}}
            \end{equation}
            which is, by definition, weight decay where 
            \begin{equation}
            \lambda_w = \eta \lambda_{\ell_2}
            \end{equation}
            
            Therefore, if we want to mimic the effect of weight decay using $\ell_2$-regularized loss function $L^{reg}_{\vec{\theta}}(\cdot, \cdot)$, we can do so by setting the weight decay factor $\lambda_w$ equal to the product of the $\ell_2$ penalty factor $\lambda_{\ell_2}$ and the learning rate $\eta$.
        }
        
        %Part A (ii)
        \qitem{ 
        We saw that $\ell_2$-regularized loss function operating in SGD can have weight-decaying effect under a constraint where $\lambda_w = g(\lambda_{\ell_2}, \eta, ...)$ for some function $g$. Since the weight decaying factor $\lambda_w$ is dependent on the learning rate $\eta$, we say these two parameters are \textit{coupled}. In other words, for some optimal weight decaying factor $\lambda_w$, we would need to scale the learning rate $\eta$ and $\ell_2$ penalization term $\lambda_{\ell_2}$ correctly to parameterize in SGD. Therefore, we are constraining our search in hyperparameter space as one is coupled with the other. 
        
        Now, we introduce the idea of \textit{weight decoupling} - a technique to decouple the relationship between $\lambda_w$ and $\eta$ in our code implementation. The weight decoupled SGD algorithm with weight decay is known as \textit{SGDW}. SGDW allows us to search through both hyperparameter space independently. \textbf{Given the SGD algorithm with momentum optimizing an $\ell_2$-regularized loss function $L_{\vec{\theta}}^{reg}(\cdot, \cdot)$ (Algorithm 1), what modification(s) should be made to turn it into SGDW with momentum for some weight decay factor $\lambda_w$?} 
        
        \textit{HINT: Refer to (\ref{weight_decay}) and ask yourself if this equation have any dependence on $\eta$.}
        
        \textit{NOTE: Keep in mind that this question additionally incorporates momentum. Therefore, be sure to also decouple hyperparameters that are used in momentum as well.}
        \begin{algorithm}\label{SGD}
            \caption{SGD with momentum optimizing $\ell_2$ regularized loss function $L_{\vec{\theta}}^{reg}(\cdot, \cdot)$}
            \begin{algorithmic}[1]
                \State \textbf{given} learning rate $\eta \in \mathbb{R}_{+}$, $\ell_2$ regularization factor $\lambda_{\ell_2} \in \mathbb{R}_{+}$, momentum factor $\beta_1 \in (0, 1]$, and data set $\{\vec{x}_i, y_i\}_{i = 1}^n$.
                \State \textbf{initialize} time step $t \leftarrow  0$, parameter vector $\vec{\theta}_0$, first moment (momentum) vector $\vec{m}_0 \leftarrow \vec{0}$
                \While{\textit{stopping criterion not met}}
                    \State $t \leftarrow t+1$
                    \State $ \vec{g}_t \leftarrow \grad L^{reg}_{\vec{\theta}}(y_i, f_{\vec{\theta}}(\vec{x}_i))\bigg|_{\vec{\theta} = \vec{\theta}_{t-1}}$
                    \Comment{ $(\vec{x}_i, y_i)$ is the random point}
                    \State $\vec{m}_t \leftarrow (1-\beta_1) \vec{m}_{t-1} + \beta_1 \vec{g}_t$
                    \State $\vec{\theta}_t \leftarrow \vec{\theta}_{t-1} - \eta \vec{m}_t$
                \EndWhile
                \State \textbf{return} optimized parameters $\vec{\theta}_t$
            \end{algorithmic} 
        \end{algorithm}
        }
        \\
        
        
        %SOL - Part A (ii)
        \sol{ \\
        To convert SGD with $\ell_2$ regularized loss function $L_{\vec{\theta}}^{reg}(\cdot, \cdot)$ into SGDW, we modify the following lines:
        \begin{itemize}
            \item \textbf{5}:  $ \vec{g}_t \leftarrow \grad L_{\vec{\theta}}(y_i, f_{\vec{\theta}}(\vec{x}_i))\bigg|_{\vec{\theta} = \vec{\theta_{t-1}}}$
            \item \textbf{7}: $\vec{\theta}_t \leftarrow \vec{\theta}_{t-1} - \eta \vec{m}_t \underline{- \lambda_{w} \vec{\theta}_{t-1}} = (1 - \lambda_w) \vec{\theta}_{t-1} - \eta \vec{m}_t$
        \end{itemize}
        
        Line \textbf{7} is like so because with $L_{\vec{\theta}}^{reg}(\cdot, \cdot)$, grouping line \textbf{7} in terms of $\vec{\theta}_{t-1}$ on the RHS gives us 
        \begin{equation}
        \vec{\theta}_t \leftarrow \vec{\theta}_{t-1} - \eta \vec{m}^{reg}_t = \vec{\theta}_{t-1} - \eta (\vec{m}_t + \beta_1 \lambda_{\ell_2} \vec{\theta}_{t-1}) = (1 - \eta \beta_1 \lambda_{\ell_2}) \vec{\theta}_{t-1} -\eta\vec{m}_t
        \end{equation}
        where $\vec{m}^{reg}_t$ is the moment vector of $L_{\vec{\theta}}^{reg}(\cdot, \cdot)$ at $t$ and $\vec{m}_t$ is such for $L_{\vec{\theta}}(\cdot, \cdot)$. We can then substitute $\lambda_w \coloneqq \eta\beta_1\lambda_{\ell_2}$ in line \textbf{8} to decouple these parameters from each other.
        }
        
    \end{enumerate}}
    
    %Part B
    \qitem{ \textbf{Weight Decay $\neq \ell_2$-Regularization in Adaptive Gradient Methods (and Hopefully, Adam)}
    
    We'll now try to prove that $\ell_2$-regularization does not give the same effect as weight decay under Adaptive Gradient Methods such as AdaGrad and RMSProp. Unfortunately, Adam cannot fall under this proof entirely due to the algorithm's additional complexity in containing momentum and using the bias-corrected momentum term for weight update. But Adam, containing adaptive methods in its core structure, can get very close to a convincing argument that weight decay $\neq \ell_2$-regularization in which we'll summarize in the problem statement of Question 2. 
    
    \begin{enumerate}[(i)]
    
        %Part B (i)
        \qitem{
        Below is the pseudo-code for the \href{https://pytorch.org/docs/stable/generated/torch.optim.RMSprop.html}{RMSProp} algorithm with bias-correcting second moment. Such is an adaptive algorithm made as close to Adam as possible while retaining its ability to prove the inequivalence between $\ell_2$-regularization and weight decay.
        
        \begin{algorithm}\label{RMSProp}
            \caption{Root Mean Square Propagation (RMSProp) with bias-correcting second moment}
            \begin{algorithmic}[1]
                \State \textbf{given} learning rate $\eta \in \mathbb{R}_{+}$, small value $\epsilon$, $\vec{\theta} \in \mathbb{R}^m$, second moment factor $\beta_2 \in (0, 1]$, and data set $\{\vec{x}_i, y_i\}_{i = 1}^n$.
                \State \textbf{initialize} time step $t \leftarrow  0$, parameter vector $\vec{\theta}_{0}$, exponential average second moment vector $\vec{v}_{0} \leftarrow \vec{0}$
                \While{\textit{stopping criterion not met}}
                    \State $t \leftarrow t+1$
                    \State $ \vec{g}_t \leftarrow \grad L_{\vec{\theta}}(y_i, f_{\vec{\theta}}(\vec{x}_i))\bigg|_{\vec{\theta} = \vec{\theta}_{t-1}}$
                    \Comment{ $(\vec{x}_i, y_i)$ is the random point}
                    \State $\vec{v}_t \leftarrow \beta_2 \vec{v}_{t-1} + (1-\beta_2)\vec{g}_t^2$
                    \Comment{$\vec{g}_t^2$ is component-wise squared of $\vec{g}_t$}
                    \State $\hat{\vec{v}}_t \leftarrow \vec{v}_t / (1-\beta_2^{t})$
                    \Comment{Bias-correcting second-moment; $\beta_2$ is taken to the power of $t$}
                    \State $\vec{\theta}_t \leftarrow \vec{\theta}_{t-1} - \eta [\vec{g}_t / (\sqrt{\hat{\vec{v}}_t} + \epsilon \vec{\mathds{1}}_m)]$
                    \Comment{Division and $\sqrt{\hat{\vec{v}}}_t$ here are component-wise operations. }
                \EndWhile
                \State \textbf{return} optimized parameters $\vec{\theta}_t$
            \end{algorithmic} 
        \end{algorithm}
        
        Adaptive methods follow a general update rule of the form:
        \begin{equation}\label{adaptive}
        \vec{\theta}_{t} \leftarrow \vec{\theta}_{t-1} - \eta \vec{M}_{t-1} \vec{g}_{t-1}
        \end{equation}
        where $\vec{M}_t \in \mathbb{R}^{m \times m}$ and $\vec{g}_t \in \mathbb{R}^{m}$ are variables that change with time $t$. 
        
        Consider a function $\vec{G}_t : \mathbb{R}_{(0, 1]} \rightarrow \mathbb{R}^{m \times m}$ that takes the exponential average (factor of $\alpha \in (0, 1]$) of the all previous gradients' outer-products:
        \begin{equation}
        \vec{G}_t(\alpha) = \sum_{\tau = 1}^{t} \alpha^{t-\tau}\vec{g}_\tau \vec{g}_\tau^T
        \end{equation}
        \textbf{For the RMSProp algorithm with bias-correcting second moment, explicitly determine what the matrix $\vec{M}_t$ is in terms of the function $\vec{G}_t(\alpha)$ and the second moment factor $\beta_2$. For simplicity, assume that $\vec{v}_0 = 0$, $\epsilon = 0$, and $\vec{g}_t \in \mathbb{R}^m$.}
        
        \textit{HINT: Recall a couple of component-wise operations can be written as: 
        \begin{itemize}
        \item $\vec{x}^2 = \vec{x} \odot \vec{x} = diag(\vec{x}\vec{x}^T)$
        \item $\vec{x} \odot \vec{y} = diag(\vec{x})\vec{y} = diag(\vec{y})\vec{x}$
        \item $\vec{x}^p = \vec{x} \odot ... \odot \vec{x} = diag(diag(\vec{x}^p)) = diag(diag(\vec{x})^p)$
        \end{itemize}
        for $\vec{x}, \vec{y} \in \mathbb{R}^n$ where $diag(\cdot)$, when having a matrix input, will vectorize all the diagonal elements. For a vector input, will construct a diagonal matrix where each diagonal containing each element of the vector. (Similar to the \texttt{numpy.diag} function)
        }
        }
        
        %Sol - Part B (i)
        \sol{
        Rewriting line \textbf{8} of the algorithm like equation (\ref{adaptive}), gives us:
        \begin{equation}
            \vec{\theta}_t \leftarrow \vec{\theta}_{t-1}-\eta \vec{\hat{v}}_t^{-\frac{1}{2}}\odot \vec{g}_t = \vec{\theta}_{t-1}-\eta diag(\vec{\hat{v}}_t^{-\frac{1}{2}})\vec{g}_t
        \end{equation}
        Hence, $\vec{M_t} = diag(\vec{\hat{v}}_t^{-\frac{1}{2}})$. 
        
        Now solving for a closed form for $\vec{v}_t$, we get
        \begin{equation}
            \vec{v}_1 = \beta_2\vec{v}_0 + (1 - \beta_2)\vec{g}_1^2 = (1 - \beta_2)\vec{g}_1^2
        \end{equation}
        \begin{equation}
            \vec{v}_2 = \beta_2(1 - \beta_2)\vec{g}_1^2 + (1 - \beta_2)\vec{g}_2^2
        \end{equation}
        \begin{equation}
            \vec{v}_3 = \beta_2^2(1 - \beta_2)\vec{g}_1^2 + \beta_2(1 - \beta_2)\vec{g}_2^2 + (1 - \beta_2)\vec{g}_3^2
        \end{equation}
        \begin{equation}
            \vec{v}_t = (1-\beta_2)\sum_{\tau = 1}^t \beta_2^{t-\tau}\vec{g}_\tau^2
        \end{equation}
        Simplifying $\vec{v}_t$ gives us:
        \begin{equation}
            \vec{v}_t = (1-\beta_2)\sum_{\tau = 1}^t \beta_2^{t-\tau}diag(\vec{g}_\tau \vec{g}_\tau^T) = (1-\beta_2)diag(\sum_{\tau = 1}^t \beta_2^{t-\tau}\vec{g}_\tau \vec{g}_\tau^T)
        \end{equation}
        Substituting our definition of $\vec{G}_t(\alpha)$,
        \begin{equation}
            \vec{v}_t = (1-\beta_2)diag(\vec{G}_t(\beta_2))
        \end{equation}
        This gives our expression of $\hat{\vec{v}}_t$ as:
        \begin{equation}
            \hat{\vec{v}}_t = \vec{v}_t / (1 - \beta_2^t) = \frac{diag(\vec{G}_t(\beta_2))}{\sum_{\tau = 0}^{t-1}\beta_2^\tau}
        \end{equation}
        Therefore, giving us
        \begin{equation}
        \vec{M}_t = \frac{diag(\vec{G}_t(\beta_2)_{(1, 1)}^{-\frac{1}{2}}, \vec{G}_t(\beta_2)_{(2, 2)}^{-\frac{1}{2}}, ... \vec{G}_t(\beta_2)_{(m, m)}^{-\frac{1}{2}} )}{\sum_{\tau = 0}^{t-1}\beta_2^\tau}
        \end{equation}
        \begin{equation}
            =\frac{1}{\sum_{\tau = 0}^{t-1}\beta_2^\tau}\begin{bmatrix}
            \vec{G}_t(\beta_2)_{(1, 1)}^{-\frac{1}{2}} & 0 & ... & 0\\
            0 & \vec{G}_t(\beta_2)_{(2, 2)}^{-\frac{1}{2}} & ... & 0\\
            ... & ... & ... & ...\\
            0 & 0 & ... & \vec{G}_t(\beta_2)_{(m, m)}^{-\frac{1}{2}} 
            \end{bmatrix}
        \end{equation}
        }
        
        %Part B (ii)
        \qitem{
        
        Now that we've showed that Algorithm 2 also satisfies equation (\ref{adaptive}) where $\vec{g}_t$ is the gradient of the loss, we can proceed with our proof. \textbf{Prove that for $\vec{M}_t \neq k\vec{I}$ for some constant $k \in \mathbb{R}$, there exists no $\ell_2$-coefficient $\lambda_{\ell_2}$ such that running adaptive gradient methods through the loss function $L_{\vec{\theta}}^{reg}(\cdot, \cdot)$ would give the same effect as weight decay.}
        }
        
        %Sol - Part B (ii)
        \sol{
        Refer to the previous part for derivation of the gradient of $L_{\vec{\theta}}^{reg}(\cdot, \cdot)$.
        
        The gradient update for Adaptive Methods with $\ell_2$ regularization is: 
        \begin{equation}
        \vec{\theta}_{t+1} \leftarrow \vec{\theta}_t - \eta \lambda_{\ell_2} \vec{M}_t \vec{\theta}_t - \eta \vec{M}_t \grad L_{\vec{\theta}} (y_i, f_{\vec{\theta}}(\vec{x}_i)) \bigg|_{\vec{\theta} = \vec{\theta}_t}
        \end{equation}
        \begin{equation}
            = (\vec{I} - \eta\lambda_{\ell_2}\vec{M}_t) \vec{\theta}_t - \eta \vec{M}_t \grad L_{\vec{\theta}} (y_i, f_{\vec{\theta}}(\vec{x}_i)) \bigg|_{\vec{\theta} = \vec{\theta}_t}
        \end{equation}

        The gradient update for Adaptive Method with Weight Decay is:
        \begin{equation}
        \vec{\theta}_{t+1} \leftarrow (\vec{I} - \lambda_w \vec{I}) \vec{\theta}_t - \eta \vec{M}_t \grad L_{\vec{\theta}}(y_i, f_{\vec{\theta}}(\vec{x}_i)) \bigg|_{\vec{\theta} = \vec{\theta}_t}
        \end{equation}

        In order for these update rules to be equivalent, we must have $\lambda_w \vec{I}= \eta \lambda_{\ell_2} \vec{M}_t$. This implies that $\vec{M}_t = \frac{\lambda_w}{\eta \lambda_{\ell_2}} \vec{I} = k \vec{I}$ for some $k \in \mathbb{R}$. Therefore, there is no $\lambda_{\ell_2}$ term that can have a weight decaying effect for adaptive methods. Generally, $\vec{M}_t$ would not be $k \vec{I}$ throughout all time-steps as this does not take advantage of adaptive methods taking different steps in each parameter dimensions - thus for cases like this, the problem can be solved with SGD and would have the same convergence rate.
        }
    
        %Part B (iii)
        \qitem{
        Now assume $\vec{M}_t$ (i.e. known as a preconditioner matrix) is a positive-definite and symmetric matrix across all time. This is can be true as many adaptive gradient methods contains a particular expression containing $\vec{G}_t(\alpha)$ for some value of $\alpha \in (0, 1]$ (Can you see why this is can be true? Does this line up to our answer of $\vec{M}_t$ from part(b)(i)?). \textbf{Given this, show that we can scale-adjust $\ell_2$-regularization in our loss function (dependent on time) like below to have a weight decaying effect: }
        
        \begin{equation}
            L_{\vec{\theta}, t}^{special\_reg}(y_i, f_{\vec{\theta}}(\vec{x}_i)) = L_{\vec{\theta}}(y_i, f_{\vec{\theta}}(\vec{x}_i)) + \frac{\lambda_{\ell_2}}{2} ||\vec{M}_t^{-\frac{1}{2}}\vec{\theta}||_2^2
        \end{equation}
        \textit{NOTE: Observe that $L_{\vec{\theta}, t}^{special\_reg}(\cdot, \cdot)$
        is a special loss function that changes per iteration. This is unique from all the loss functions we've covered as many optimization problems only look at one loss function to minimize. Although this is something we can implement in code, you should realize that this is a special modification of our natural understanding of loss functions.
        }
        
        }
        
        %Sol - Part B (iii)
        \sol{
        Taking the gradient of $L_{\vec{\theta}}^{special\_reg}(y_i, f_{\vec{\theta}}(\vec{x}_i))$ in respect to $\vec{\theta}$ gives us:
        
        \begin{equation}
            \grad L_{\vec{\theta}}^{special\_reg}(y_i, f_{\vec{\theta}}(\vec{x}_i))\bigg|_{\vec{\theta} = \vec{\theta}_t} = L_{\vec{\theta}}(y_i, f_{\vec{\theta}}(\vec{x}_i))\bigg|_{\vec{\theta} = \vec{\theta}_t} + \lambda_{\ell_2}(\vec{M}_t^{-\frac{1}{2}})^T (\vec{M}_t^{-\frac{1}{2}})\vec{\theta}_t
        \end{equation}
        Since $\vec{M}_t$ is PD and symmetric, so is $\vec{M}_t^{-\frac{1}{2}}$,
        \begin{equation}
            \grad L_{\vec{\theta}}^{special\_reg}(y_i, f_{\vec{\theta}}(\vec{x}_i))\bigg|_{\vec{\theta} = \vec{\theta}_t} = L_{\vec{\theta}}(y_i, f_{\vec{\theta}}(\vec{x}_i))\bigg|_{\vec{\theta} = \vec{\theta}_t} + \lambda_{\ell_2}\vec{M}_t^{-1}\vec{\theta}_t
        \end{equation}
        Therefore, the update function,
        \begin{equation}
            \vec{\theta}_{t+1} = \vec{\theta}_t - \eta \vec{M}_t\grad L_{\vec{\theta}}^{special\_reg}(y_i, f_{\vec{\theta}}(\vec{x}_i))\bigg|_{\vec{\theta} = \vec{\theta}_t}
        \end{equation}
        becomes
        \begin{equation}
            \vec{\theta}_{t+1} = (1 - \eta \lambda_{\ell_2})\vec{\theta}_t - \eta \vec{M}_t\grad L_{\vec{\theta}}(y_i, f_{\vec{\theta}}(\vec{x}_i))\bigg|_{\vec{\theta} = \vec{\theta}_t}
        \end{equation}
        where $\lambda_w = \eta \lambda_{\ell_2}$ like the SGD case. 
        
        }
    \end{enumerate}
    
    
    }
    
    
    
\end{enumerate}